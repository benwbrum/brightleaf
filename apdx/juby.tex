%********** Appendix 2 **********

\chapter{Juby Towler's account of Thomas Keeling Brumfield's Marriage}


P 98

The untimely death of Sarah's husband in 1845 left her with a large family of extremely young children.  Her happiness while Henry lived now suddenly became only a memory.  The friendly, even if not warm, attitude of the big Towler clan changed immediately.  The old prejudices returned and they no longer "darkened her door."  While Henry lived ,his ability as a successful farmer and family provider demanded adn got their respect.  The stigma of his birth was fadin gfast with his growing reputation of success.  But, after his death attitudes changed.  Sarah was unable to care for her children and manage theplantation so that which Henry had laid aside was soon gone, and his family had to battle poverty and starvation.  Pride and security vanished.  Sarah and her young children were innocent.  They were victims of prejudice.  They were actually starving.

p 100

Perhaps others nearby sat down that Christmas day to a table lavishly burdened with food and thinking of tese innocent children may have adhered to an old adage that originated when an ancient city was being besieged.  "Let them all starve, God will know his own."

Early in January of 1847 Sarah had a visitor.  he rode up on horseback, a quiet but gentle looking man. He was tall, rode easy in the saddle, and his calm facial expression was that of strong determination but kind and thoughtful.  He dismounted slowly, tied his horse to a tree limb, and came to her front door.  His name was Thomas K. Brumfield.

A kind providence had intervened for Sarah Towler and her children.  Her plight had reached the ears of Thomas, about 30 miles east.  he, too, had lost his mate.  He, too, had 8 young children.  He came to Sarah with a proposal of marriage.  She needed a husband and a fathe rfor her 8 children, and he needed a wife and a mother for his 8 children.  Both had 4 daughters and 4 sons.  Both had a desperate need the other could fill.  Sarah Towler married Thomas K. Brumfield on Feb. 22, 1847, and now they had a family of 16 children the oldest of whom was only 13 years old.

Whatever specific agreement, if any, was made between Thomas and Sarah; and whatever secrets that dwelled in their hearts, from memories past or between them now were interred with their bodies.  Theirs may not have been a marriange union in passionate love, for they never had any children mutually their own, but their union was an infinitely happy one; and hwo can say more than, "No man hath greater love than that he will lay down his life for a friend."

From the very moment Sarah and Thomas brought their children together as a family they were roaringly happy.  Many years later an aged member



p 102

day when they were standing in the front door of the family home and while the entire family listened and urged them on.  After their marriage they lived on the plantation until mother Sarah died.  In 1873 they packed up and permanently moved away from Va.  They settled near Patoka, Illinois on the same farm formerly settled by Joseph and Henry Towler and Josh Brumfield.  John had a large family part of which moved into Kansas and some of the others went south into Arkansas near Litle Rock.

[section on No. 2 -- nancy omitted]

After the family began splitting up as they reached adult age Thomas Brumfield began selling off parts of the large plantation.  Very little is known of the Brumfield children except the oldest


[Towler children summary continues through p 106]

%********** End of chapter **********
