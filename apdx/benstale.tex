%********** Appendix 2 **********

\chapter{Memoir of B. F. Brumfield}

Just a few changes in sixty years as observed by one man. 

I was born on Nov. 8 1886, the 14th child of Henry A. and Julia Craddock Brumfield. That was just ahead of the gloomy days of Cleveland's administration, before the days when it became a disgrace to have a large family. I am told that that was an exceedingly wet year ruining most crops so no doubt my arrival was considered just one more downfall. But in those days it was not a matter of so much expense. No doubt there was plenty of hand-me-downs for a layette, and if a doctor attended at all which did not always happen it was not a case of over a \$3.00 fee. If there was extra help needed in the house that could be secured easily and at a cost of not over \$8.00 per month.  

Those large families were an asset then for all worked to one end, a living for all. One of my earliest recollections was my first pair of pants. Was I proud of them. Most boys wore dresses until they were about four years old. So when I was between three and four I was dressed out in home-made pants. I promptly managed to fall in the mud and they were taken off. I guess my head was so high with pride that I failed to see the ground. I still remember that as a great calamity. Another early recollection is a mowing machine my father bought. That was the first one bought in this vicinity. It is said my father sowed the first hay raised in this community and this was cut with hand mowing blades and raked with pitch-forks. If some of the old ones could see the changes in the farming they could not believe their eyes. 

One of the great problems even to that time was to remove the wood from the land and clear it off for production. Millions of feet of original pine and oak was cut and burned to get rid of it. Now it would be valuable indeed. I have helped keep the fire going in logs for a week or more. If the farmers could have seen the results of that destruction and left the wood on all steep acres our land would not be so eroded now. Another big drawback was lack of tools to work with. Very few owned a two-horse plow and they were small. The crop was plowed with a  single or at most a double shovel plow. That made a small part of the row and left most of the weeds and grass to be chopped by hand. That meant about four hoe hands to each horse plowing. Those hands cost about 25 or 30 cents a day to those who had to hire help. No wonder they were willing to raise their own help! The soil was  only scratched shallow  with this cultivation  and with no winter cover of course it was soon gone and more had to be cleared.

I do not think the women worked in the fields more than they do now as there was so much to be done at the house. Spinning and weaving was beginning to be left off but the women cut and made their own clothes as well as the work clothes for the men and children. All of the bed clothes were hand made and some of the cotton and wool home raised. Our job for rainy days was to pick the burrs and trash from wool and the seed from cotton. 

Most of the dresses were made of cotton plaids or calico then. It took ten yards of calico to make a  ladies dress. When it was made with Leg-O-Mutton sleeves, wasp waist and skirt sweeping the floor it was very stylish. I might add that this calico would cost about five cents a yard. We had the same Clarks thread on the market then. 200 yds O.N.T. cotton 5 cents. I can remember the ladies dress shoes. Button style. Cost from about \$1.25 to \$2.50. 

Some of the mens styles in shoes were garters. Later we had the tooth-pick toes. Bunion and corn builders. But they were stylish.Oh! I must not forget the Celluloid collars and cuff for men. I guess there were no bath-tubs so it was necessary to wear clothes that you could sponge off with a wet rag. Most of the mens ties were a ready tied cravat with a very large knot. See old photos. I took no part in any of this as I was small and stayed at home. The first ready made clothes I ever owned  was a long-pants suit and then that was never worn until most grown. My first dress shoe was No. 8. 

When I was five years old I was allowed to go to school for my first time. The teacher must have had trouble making an average for she sent word for me to come back. So that's where trouble started for me. Our free schools were usually five month terms taught by some bright young lady that had completed seventh grade work or some man with a fair education and so opposed to white labor that he wouldn't do anything else.  It used to be said his next step was Legislature. I think the average teacher recd. about twenty dollars per month. 

We studied the old McGuffey Readers and perhaps that is why some of the pupils of that day made the men and women they did. All of those books taught a moral. More or less Greek to the child but coming back to you in later years.

Another great change was the social side of the working people's life. This was the days of the horse and buggy. If the average young man possessed a decent looking suit and outfit and could get \$20 or \$25.00 pocket money he could make out alright. The boys had to work most of the time. An occasional dance and a get together on Sat night, Sunday School or church on Sun. was about all the time he had off. 

Most of the boys made music on something (or tried) so they had no orchestra hire. If they went in for liquid refreshments the cost was 25 cents a pint! Of course some went in for that but it was so cheap it was not half so attractive. I think perhaps the young was happier than today. The bright lights were not so attractive and if they had been as bright we had no way of seeing them so we were content at home. Marriages held better as a rule. The couple had generally known each other better and did not expect so much of life. Today life is pictured so rosy and so little is required of the young that the period of disillusionment is so great that few of the young can take it. I think that the picture placed before them by fiction, radio and television misleads many to expect too much.     

\textit{This document was found in an old desk at the Brumfield home place in Renan and was written by B. F. Brumfield. It was written in pencil on the back of 10 sheets of paper that were hand written ink lists of dahlias and their properties. I would suppose he wrote this some time in the late 1940's based on his being sixty years old and regarding television as an enticement. I wonder what he'd say about today's television. The dahlia list had three columns which listed name, size and kind, and color. The list started with Avalon, A.F.D., Bright yellow and ended with Yellow Pom, ball, yellow. These appeared to have been copied from a reference book.}

\textit{Neil Brumfield}

\textit{Bridge City, Texas 1994}

